% Options for packages loaded elsewhere
% Options for packages loaded elsewhere
\PassOptionsToPackage{unicode}{hyperref}
\PassOptionsToPackage{hyphens}{url}
\PassOptionsToPackage{dvipsnames,svgnames,x11names}{xcolor}
%
\documentclass[
  letterpaper,
  DIV=11,
  numbers=noendperiod,
  oneside]{scrartcl}
\usepackage{xcolor}
\usepackage[margin=1in]{geometry}
\usepackage{amsmath,amssymb}
\setcounter{secnumdepth}{-\maxdimen} % remove section numbering
\usepackage{iftex}
\ifPDFTeX
  \usepackage[T1]{fontenc}
  \usepackage[utf8]{inputenc}
  \usepackage{textcomp} % provide euro and other symbols
\else % if luatex or xetex
  \usepackage{unicode-math} % this also loads fontspec
  \defaultfontfeatures{Scale=MatchLowercase}
  \defaultfontfeatures[\rmfamily]{Ligatures=TeX,Scale=1}
\fi
\usepackage{lmodern}
\ifPDFTeX\else
  % xetex/luatex font selection
\fi
% Use upquote if available, for straight quotes in verbatim environments
\IfFileExists{upquote.sty}{\usepackage{upquote}}{}
\IfFileExists{microtype.sty}{% use microtype if available
  \usepackage[]{microtype}
  \UseMicrotypeSet[protrusion]{basicmath} % disable protrusion for tt fonts
}{}
\makeatletter
\@ifundefined{KOMAClassName}{% if non-KOMA class
  \IfFileExists{parskip.sty}{%
    \usepackage{parskip}
  }{% else
    \setlength{\parindent}{0pt}
    \setlength{\parskip}{6pt plus 2pt minus 1pt}}
}{% if KOMA class
  \KOMAoptions{parskip=half}}
\makeatother
% Make \paragraph and \subparagraph free-standing
\makeatletter
\ifx\paragraph\undefined\else
  \let\oldparagraph\paragraph
  \renewcommand{\paragraph}{
    \@ifstar
      \xxxParagraphStar
      \xxxParagraphNoStar
  }
  \newcommand{\xxxParagraphStar}[1]{\oldparagraph*{#1}\mbox{}}
  \newcommand{\xxxParagraphNoStar}[1]{\oldparagraph{#1}\mbox{}}
\fi
\ifx\subparagraph\undefined\else
  \let\oldsubparagraph\subparagraph
  \renewcommand{\subparagraph}{
    \@ifstar
      \xxxSubParagraphStar
      \xxxSubParagraphNoStar
  }
  \newcommand{\xxxSubParagraphStar}[1]{\oldsubparagraph*{#1}\mbox{}}
  \newcommand{\xxxSubParagraphNoStar}[1]{\oldsubparagraph{#1}\mbox{}}
\fi
\makeatother


\usepackage{longtable,booktabs,array}
\usepackage{calc} % for calculating minipage widths
% Correct order of tables after \paragraph or \subparagraph
\usepackage{etoolbox}
\makeatletter
\patchcmd\longtable{\par}{\if@noskipsec\mbox{}\fi\par}{}{}
\makeatother
% Allow footnotes in longtable head/foot
\IfFileExists{footnotehyper.sty}{\usepackage{footnotehyper}}{\usepackage{footnote}}
\makesavenoteenv{longtable}
\usepackage{graphicx}
\makeatletter
\newsavebox\pandoc@box
\newcommand*\pandocbounded[1]{% scales image to fit in text height/width
  \sbox\pandoc@box{#1}%
  \Gscale@div\@tempa{\textheight}{\dimexpr\ht\pandoc@box+\dp\pandoc@box\relax}%
  \Gscale@div\@tempb{\linewidth}{\wd\pandoc@box}%
  \ifdim\@tempb\p@<\@tempa\p@\let\@tempa\@tempb\fi% select the smaller of both
  \ifdim\@tempa\p@<\p@\scalebox{\@tempa}{\usebox\pandoc@box}%
  \else\usebox{\pandoc@box}%
  \fi%
}
% Set default figure placement to htbp
\def\fps@figure{htbp}
\makeatother





\setlength{\emergencystretch}{3em} % prevent overfull lines

\providecommand{\tightlist}{%
  \setlength{\itemsep}{0pt}\setlength{\parskip}{0pt}}



 


\usepackage{titlesec}
\usepackage{xcolor}
\definecolor{dccgreen}{HTML}{006838}
\titleformat{\section}{\large\bfseries\color{dccgreen}}{}{0em}{}[\titlerule]
\pagenumbering{gobble} # Removes page numbers if it's a short syllabus
\KOMAoption{captions}{tableheading}
\makeatletter
\@ifpackageloaded{caption}{}{\usepackage{caption}}
\AtBeginDocument{%
\ifdefined\contentsname
  \renewcommand*\contentsname{Table of contents}
\else
  \newcommand\contentsname{Table of contents}
\fi
\ifdefined\listfigurename
  \renewcommand*\listfigurename{List of Figures}
\else
  \newcommand\listfigurename{List of Figures}
\fi
\ifdefined\listtablename
  \renewcommand*\listtablename{List of Tables}
\else
  \newcommand\listtablename{List of Tables}
\fi
\ifdefined\figurename
  \renewcommand*\figurename{Figure}
\else
  \newcommand\figurename{Figure}
\fi
\ifdefined\tablename
  \renewcommand*\tablename{Table}
\else
  \newcommand\tablename{Table}
\fi
}
\@ifpackageloaded{float}{}{\usepackage{float}}
\floatstyle{ruled}
\@ifundefined{c@chapter}{\newfloat{codelisting}{h}{lop}}{\newfloat{codelisting}{h}{lop}[chapter]}
\floatname{codelisting}{Listing}
\newcommand*\listoflistings{\listof{codelisting}{List of Listings}}
\makeatother
\makeatletter
\makeatother
\makeatletter
\@ifpackageloaded{caption}{}{\usepackage{caption}}
\@ifpackageloaded{subcaption}{}{\usepackage{subcaption}}
\makeatother
\makeatletter
\@ifpackageloaded{sidenotes}{}{\usepackage{sidenotes}}
\@ifpackageloaded{marginnote}{}{\usepackage{marginnote}}
\makeatother
\usepackage{bookmark}
\IfFileExists{xurl.sty}{\usepackage{xurl}}{} % add URL line breaks if available
\urlstyle{same}
\hypersetup{
  pdftitle={Elementary Statistics (OER) - MAT 118},
  pdfauthor={Professor Henry Mendoza Rivera},
  colorlinks=true,
  linkcolor={blue},
  filecolor={Maroon},
  citecolor={Blue},
  urlcolor={Blue},
  pdfcreator={LaTeX via pandoc}}


\title{Elementary Statistics (OER) - MAT 118}
\usepackage{etoolbox}
\makeatletter
\providecommand{\subtitle}[1]{% add subtitle to \maketitle
  \apptocmd{\@title}{\par {\large #1 \par}}{}{}
}
\makeatother
\subtitle{Spring 2026 Syllabus \textbar{} Dutchess Community College}
\author{Professor Henry Mendoza Rivera}
\date{}
\begin{document}
\maketitle


\subsection{Course Description and
Purpose}\label{course-description-and-purpose}

Satisfies the mathematics requirement of the associate in Arts degree
program. Basic statistical procedures are developed. Topics include
descriptive statistics, hypothesis testing and confidence intervals and
regression using both simulation and a theory-based approach. Technology
will be used regularly throughout the course.

\textbf{Pre-requisites and/or co-requisites:} Placement level 2 (see DCC
math placement table), OR ENG101 placement level or higher, OR High
School GPA of 3.0 (83) or higher.

\subsection{Course Learning Outcomes}\label{course-learning-outcomes}

Students who successfully complete the course will be able to:

\begin{enumerate}
\def\labelenumi{\arabic{enumi}.}
\tightlist
\item
  Calculate and use descriptive statistics to analyze a data set
  (measures of center, measures of variation, distribution shapes,
  outliers).
\item
  Understand probability as a long-run relative frequency and interpret
  a p-value in terms of probability.
\item
  Use simulation and theory-based approaches like the Central Limit
  Theorem to draw inferences and reach conclusions.
\item
  Construct and interpret confidence intervals for a population mean or
  proportion.
\item
  Use hypothesis testing to test a claim and interpret results about a
  population mean or proportion.
\item
  Use technology to conduct simulations and generate values used in
  statistical inferences.
\item
  Use simulation and theoretical approaches to perform correlation and
  regression analysis and interpret the results.
\end{enumerate}

\subsection{Institutional Student Learning
Outcomes}\label{institutional-student-learning-outcomes}

\begin{itemize}
\tightlist
\item
  \textbf{ISLO \#4: Quantitative Reasoning.} Students will work with
  graphical, numerical, or symbolic models to solve problems and
  interpret results.
\end{itemize}

\subsection{Instructor Contact \& Office
Hours}\label{instructor-contact-office-hours}

\begin{itemize}
\tightlist
\item
  \textbf{Office:} Washington 114 \textbar{} \textbf{Phone:}
  845-431-8554
\item
  \textbf{Email:}
  \href{mailto:henry.mendozarivera@sunydutchess.edu}{\nolinkurl{henry.mendozarivera@sunydutchess.edu}}(email
  is the best way to contact me). I strive to respond to emails
  promptly, but please note that I may wait to respond. If you have not
  heard from me within 24 hours, please follow up with a reminder.
\end{itemize}

\subsubsection{Instructor Office Hours (for help, tutoring, questions,
etc)}\label{instructor-office-hours-for-help-tutoring-questions-etc}

I am available by appointment for both online and in-person meetings
during office hours. To schedule a meeting, please visit
\textbf{Brightspace}, navigate to the \textbf{``Getting Started''}
section, and click the \textbf{``Office Hours Appointment''} link under
Student Resources. You'll also find the Zoom link for online
appointments.

If the available times don't suit you, please message or email me to
arrange a different time that works for both of us. \textbf{Drop-ins are
permitted.}

\textbf{Drop-in Hours (Math \& Science Center):}

\begin{longtable}[]{@{}lll@{}}
\toprule\noalign{}
Day & Time & Location \\
\midrule\noalign{}
\endhead
\bottomrule\noalign{}
\endlastfoot
\textbf{MONDAYS} & 2:30 p.m. -- 4:00 p.m. & Washington 126 \\
\textbf{TUESDAYS} & 3:30 p.m. -- 4:30 p.m. & Washington 126 \\
\textbf{WEDNESDAYS} & 2:30 p.m. -- 4:00 p.m. & Washington 126 \\
\end{longtable}

\subsubsection{Communication Policy}\label{communication-policy}

\begin{itemize}
\tightlist
\item
  \textbf{Official Email:} The college-supplied email class list will be
  used to share all critical information. It is imperative that your
  \textbf{@sunydutchess.edu} email is active and checked regularly.
\item
  \textbf{Subject Line:} Students should contact via email using the
  prefix \textbf{``MAT118-section number''} as the email subject header.
\end{itemize}

\subsubsection{Purpose of Office Hours}\label{purpose-of-office-hours}

Office hours are intended to provide a time for students to:

\begin{itemize}
\tightlist
\item
  \textbf{Clarify concepts:} Discuss unclear points from lectures or
  readings.
\item
  \textbf{Get feedback:} Discuss homework assignments or projects and
  receive feedback on progress.
\item
  \textbf{Explore ideas:} Engage in deeper discussions about course
  topics.
\item
  \textbf{Build relationships:} Get to know the instructor and ask
  questions about the course or career paths.
\end{itemize}

\subsubsection{How to Prepare}\label{how-to-prepare}

To make the most of office hours, students are encouraged to:

\begin{itemize}
\tightlist
\item
  \textbf{Come prepared:} Review class materials and attempt homework
  assignments before visiting.
\item
  \textbf{Identify specific questions:} Have clear questions or areas of
  confusion ready to discuss.
\item
  \textbf{Communicate challenges:} If you are struggling, please let me
  know. We can work together to find effective solutions and support
  resources.
\end{itemize}

\subsection{Course Access
(Brightspace)}\label{course-access-brightspace}

To access the reading and videos for the course: You will be required to
use Brightspace throughout the semester. Readings, videos, assignments,
and tests will be found in Brightspace and you will turn in work through
Brightspace. \textbf{Access Link:}
\url{https://mylearning.suny.edu/d2l/home}

\subsection{Required Course Materials}\label{required-course-materials}

\begin{itemize}
\tightlist
\item
  \textbf{Textbook:} OER materials (electronic in rightspace). Printed
  copies at DCC Bookstore.
\item
  Online access to Brightspace: Go to
  \url{https://mylearning.suny.edu/d2l/home}
\end{itemize}

\subsection{Grading \& Assignments}\label{grading-assignments}

\begin{longtable}[]{@{}ll@{}}
\toprule\noalign{}
Category & Weight \\
\midrule\noalign{}
\endhead
\bottomrule\noalign{}
\endlastfoot
\textbf{Exams} (Exam 1, 2, and 3) & 45\% (15\% each) \\
\textbf{Cumulative Final Exam} & 15\% \\
\textbf{Computer Homework} (21 assignments, 2 drops) & 10\% \\
\textbf{Written Homework} (8 assignments, 1 drop) & 8\% \\
\textbf{In-class Quizzes} (4 quizzes, 1 drop) & 8\% \\
\textbf{Computer Quizzes} (22 quizzes, 2 drops) & 5\% \\
\textbf{In-class Activities} (5 activities, 1 drop) & 5\% \\
\textbf{Exam Corrections \& Discussion} & 3\% \\
\textbf{Exit Tickets} (2 drops) & 1\% \\
\end{longtable}

\begin{figure*}

See the tentative calendar of due dates for the assignments for the
entire semester in Brightspace. The calendar also lists the topics
covered in class each day of the semester.

\end{figure*}%

\subsection{Tentative Final Exam
Schedule}\label{tentative-final-exam-schedule}

\begin{longtable}[]{@{}
  >{\raggedright\arraybackslash}p{(\linewidth - 6\tabcolsep) * \real{0.2500}}
  >{\raggedright\arraybackslash}p{(\linewidth - 6\tabcolsep) * \real{0.2500}}
  >{\raggedright\arraybackslash}p{(\linewidth - 6\tabcolsep) * \real{0.2500}}
  >{\raggedright\arraybackslash}p{(\linewidth - 6\tabcolsep) * \real{0.2500}}@{}}
\toprule\noalign{}
\begin{minipage}[b]{\linewidth}\raggedright
Section
\end{minipage} & \begin{minipage}[b]{\linewidth}\raggedright
Date
\end{minipage} & \begin{minipage}[b]{\linewidth}\raggedright
Time
\end{minipage} & \begin{minipage}[b]{\linewidth}\raggedright
Location
\end{minipage} \\
\midrule\noalign{}
\endhead
\bottomrule\noalign{}
\endlastfoot
\textbf{010} & Monday, 5/11/26 & 11:00 a.m. -- 1:30 p.m. & Hudson Hall
506 \\
\textbf{020} & Wednesday, 5/6/26 & 8:00 a.m. -- 10:30 a.m. & Washington
248 \\
\textbf{030} & Wednesday, 5/6/26 & 11:00 a.m. -- 1:30 p.m. & Hudson Hall
226 \\
\textbf{110} & Tuesday, 5/12/26 & 2:00 p.m. -- 4:30 p.m. & Hudson Hall
506 \\
\end{longtable}

\begin{itemize}
\tightlist
\item
  You must go to MyDCC website to confirm the final schedule.
\end{itemize}

\subsection{How assignments and exams work in this
class}\label{how-assignments-and-exams-work-in-this-class}

\begin{itemize}
\tightlist
\item
  \textbf{Written Homework:} Students must also type all work and
  explanations for ments and write complete sentences in context.
  Students must also copy and paste screenshots from the Applet website
  into the written homework. It is essential to start as early as
  possible, put effort into these assignments, and seek help to complete
  each assignment fully and as accurately as possible. All written
  homework should be written clearly. Plagiarism is using others' ideas
  and/or words without clearly acknowledging the source of that
  information (intentional or unintentional). Cutting and pasting from a
  website is plagiarism. Copying homework is plagiarism. Any paper or
  assignment suspected of plagiarism will receive a zero.
\item
  \textbf{Computer Homework:} Computer homework problems can be
  attempted repeatedly up to the due date. Many of the problems include
  video help or similar resources. After the due date passes, students
  can access those homework problems in ``review mode,'' which means
  that the problems can be practiced, but the scores won't count in the
  gradebook.
\item
  \textbf{Computer Quizzes:} Computer quizzes contain a small selection
  of problems from the computer homework, and students only get one
  attempt at the quiz. The quizzes are timed, and no help features are
  available during the quizzes. Students must complete quizzes
  independently without accessing other resources or working with
  others.
\item
  \textbf{In-class Quizzes:} In-class quizzes will be written on paper
  and take place during the class period. The quizzes will assess
  concepts from the class or previous classes.
\item
  \textbf{In-class Activities:} Students will solve in-class activities
  during the class. These activities can be: Discussion-based activities
  (think-pair-share): Students individually reflect on a question or
  topic, discuss their thoughts with a partner, and finally share their
  ideas with the larger group.
\end{itemize}

Problem-solving activities: Students analyze real-world scenarios and
apply their knowledge to develop solutions (case studies). Also,
students take on different roles and simulate real-life situations, such
as a job interview or a negotiation (role-playing).

Creative activities: short writing assignments (writing exercises).
Students present their research, projects, or ideas to the class
(presentations).

Technology-based activities: online quizzes and polls, interactive
simulations, and collaborative online activities using tools like Google
Docs.

\begin{itemize}
\item
  \textbf{Exam corrections (Exam 1, Exam 2, and Exam 3):} You should
  write on printer paper your own correction of each question of each
  chapter exam and submit a pdf file in the place posted in Brightspace
  the next class after you receive your exam grade paper.
\item
  \textbf{Exam corrections discussion (Exam 1, Exam 2, and Exam 3):} You
  must set an office hours appointment in the fol- lowing two weeks of
  the exam to discuss one exam correction (Exam 1, Exam 2, or Exam 3)
  with the instructor.
\item
  \textbf{Exit tickets:} At the end of each class, you should answer the
  questions asked in the exit ticket survey posted in Brightspace under
  the class topic section.
\item
  \textbf{Chapter Exams:} Exams in this class take place in the
  classroom and will be written on paper.
\item
  \textbf{Cumulative Final Exam:} This will be a written exam during the
  final class, held in the classroom.
\item
  You must go to MyDCC website to confirm the final schedule How
  assignments and exams work in this class:
\end{itemize}

\subsection{Absence and Late Work
Policy}\label{absence-and-late-work-policy}

\begin{itemize}
\tightlist
\item
  \textbf{Computer Work:} NEVER accepted late. Two drops provided.
\item
  \textbf{Written Homework:} NEVER accepted late. One drop provided.
\item
  \textbf{Exams/Quizzes:} Email instructor \textbf{on or before} the day
  of. Makeup within 48 hours.
\end{itemize}

\subsection{Academic Dishonesty \&
Integrity}\label{academic-dishonesty-integrity}

The College regards academic dishonesty in any form as a breach of
academic ethics. Violation results in a zero; a second offense results
in an F for the course.

\subsection{Study Time \& Conduct}\label{study-time-conduct}

You are expected to spend at least \textbf{nine hours a week} studying.
Read actively, master formulas, and arrive to class on time. Arrange
personal business outside of class hours.

\subsection{Tentative Course Outline
Schedule}\label{tentative-course-outline-schedule}

\begin{longtable}[]{@{}ll@{}}
\toprule\noalign{}
Chapter & Topic \\
\midrule\noalign{}
\endhead
\bottomrule\noalign{}
\endlastfoot
\textbf{01} & Introduction to Statistics and Simulation \\
\textbf{02} & Introduction to Hypothesis Testing \\
\textbf{03} & Theory-Based Stats and Comparing Two Parameters \\
\textbf{04} & Confidence Intervals \\
\textbf{05} & Correlation and Regression \\
\end{longtable}

\subsection{Academic Accommodations \& Title
IX}\label{academic-accommodations-title-ix}

\begin{itemize}
\tightlist
\item
  \textbf{OAS:} Register with the Office of Accommodative Services (Room
  103) at (845) 431-8055.
\item
  \textbf{Title IX:} Reports of sexual harassment or misconduct:
  \href{mailto:TitleIX@sunydutchess.edu}{\nolinkurl{TitleIX@sunydutchess.edu}}.
\end{itemize}

\subsection{Starfish}\label{starfish}

Starfish connects students to faculty and support services. Please check
your \textbf{myDCC email} and log into \textbf{Starfish daily}.

\subsection{Statement Regarding Health and
Wellbeing}\label{statement-regarding-health-and-wellbeing}

\textbf{YOUR WELLBEING MATTERS} The DCC Counseling Center provides
individual/group counseling to help you achieve your mental health
goals. * \textbf{988 Suicide \& Crisis Lifeline:} Call or text
\textbf{988} for 24-hour support.

\subsection{AI Policy}\label{ai-policy}

AI is an emerging technology. It is inaccurate, so students must not use
AI to try and ''fact-find'' unless they follow up with their own
legitimate research. Students in this course must not use AI in any way
to generate material for any assignment unless explicitly instructed by
the teacher to do so.




\end{document}
